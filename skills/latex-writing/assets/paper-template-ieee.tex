% IEEE Conference/Transaction Paper Template
% Research Skills Plugin - LaTeX Writing Skill
%
% For IEEE Transactions, use: \documentclass[journal]{IEEEtran}
% For IEEE Conferences, use: \documentclass[conference]{IEEEtran}
%
% Compile with: pdflatex paper.tex && bibtex paper && pdflatex paper.tex && pdflatex paper.tex

\documentclass[conference]{IEEEtran}
% For journal papers, use: \documentclass[journal]{IEEEtran}

%% ============================================================================
%% PACKAGES
%% ============================================================================

% Mathematics
\usepackage{amsmath,amssymb,amsfonts}

% Graphics
\usepackage{graphicx}
\usepackage{textcomp}

% Algorithms
\usepackage{algorithmic}

% Tables
\usepackage{array}
\usepackage{multirow}

% Citations
\usepackage{cite}

% Colors (optional)
\usepackage{xcolor}

% Hyperlinks (load last)
\usepackage{hyperref}
\hypersetup{
    colorlinks=true,
    linkcolor=black,
    citecolor=black,
    urlcolor=blue
}

%% ============================================================================
%% CUSTOM COMMANDS
%% ============================================================================

\newcommand{\R}{\mathbb{R}}
\newcommand{\E}{\mathbb{E}}
\newcommand{\argmax}{\operatorname{argmax}}
\newcommand{\argmin}{\operatorname{argmin}}

%% ============================================================================
%% DOCUMENT METADATA
%% ============================================================================

\title{Your Paper Title Here\\
{\footnotesize Subtitle if needed}}

\author{
    \IEEEauthorblockN{First Author\IEEEauthorrefmark{1},
    Second Author\IEEEauthorrefmark{2}, and
    Third Author\IEEEauthorrefmark{1}}
    \IEEEauthorblockA{\IEEEauthorrefmark{1}Department of Computer Science\\
    University Name, City, Country\\
    Email: \{first, third\}@university.edu}
    \IEEEauthorblockA{\IEEEauthorrefmark{2}Department of Electrical Engineering\\
    Another University, City, Country\\
    Email: second@another.edu}
}

%% ============================================================================
%% DOCUMENT CONTENT
%% ============================================================================

\begin{document}

\maketitle

%% ----------------------------------------------------------------------------
%% ABSTRACT
%% ----------------------------------------------------------------------------

\begin{abstract}
    This document provides a template for IEEE conference and journal papers.
    The abstract should be 150-250 words and summarize the paper's objectives,
    methodology, key results, and conclusions. Do not include citations or
    abbreviations in the abstract.
\end{abstract}

%% ----------------------------------------------------------------------------
%% KEYWORDS
%% ----------------------------------------------------------------------------

\begin{IEEEkeywords}
    keyword1, keyword2, keyword3, keyword4, keyword5
\end{IEEEkeywords}

%% ----------------------------------------------------------------------------
%% INTRODUCTION
%% ----------------------------------------------------------------------------

\section{Introduction}
\label{sec:intro}

Introduce the problem and motivate the research. IEEE papers typically have
a dense introduction that covers:

\begin{itemize}
    \item Background and motivation
    \item Problem statement
    \item Brief literature review
    \item Main contributions (often as a bulleted list)
    \item Paper organization
\end{itemize}

The main contributions of this paper are:
\begin{enumerate}
    \item First contribution
    \item Second contribution
    \item Third contribution
\end{enumerate}

The rest of this paper is organized as follows. Section~\ref{sec:related}
reviews related work. Section~\ref{sec:method} presents our methodology.
Section~\ref{sec:experiments} describes experiments and results.
Section~\ref{sec:conclusion} concludes the paper.

%% ----------------------------------------------------------------------------
%% RELATED WORK
%% ----------------------------------------------------------------------------

\section{Related Work}
\label{sec:related}

Review relevant literature. In IEEE papers, this section is often brief and
focused. Cite papers using~\cite{ref1} or multiple citations~\cite{ref1,ref2}.

\subsection{Topic Area A}
Discuss related work in area A.

\subsection{Topic Area B}
Discuss related work in area B.

%% ----------------------------------------------------------------------------
%% METHODOLOGY
%% ----------------------------------------------------------------------------

\section{Proposed Method}
\label{sec:method}

Describe your methodology. IEEE papers often use many equations and algorithms.

\subsection{Problem Formulation}

Define the problem mathematically:
\begin{equation}
    \min_{\mathbf{x}} f(\mathbf{x}) \quad \text{subject to} \quad g(\mathbf{x}) \leq 0
    \label{eq:problem}
\end{equation}

\subsection{Algorithm Description}

Present your algorithm:
\begin{equation}
    x_{t+1} = x_t - \alpha \nabla f(x_t)
    \label{eq:update}
\end{equation}

You can also use the algorithmic environment:
\begin{algorithmic}
    \STATE Initialize $x_0$
    \FOR{$t = 1$ to $T$}
        \STATE Compute gradient $g_t = \nabla f(x_t)$
        \STATE Update $x_{t+1} = x_t - \alpha g_t$
    \ENDFOR
    \RETURN $x_T$
\end{algorithmic}

%% ----------------------------------------------------------------------------
%% EXPERIMENTS
%% ----------------------------------------------------------------------------

\section{Experiments}
\label{sec:experiments}

Present experimental results.

\subsection{Experimental Setup}

Describe datasets, metrics, and baselines.

\subsection{Results}

Present results in tables:

\begin{table}[htbp]
    \caption{Comparison of Methods}
    \label{tab:results}
    \centering
    \begin{tabular}{|l|c|c|c|}
        \hline
        \textbf{Method} & \textbf{Accuracy} & \textbf{F1} & \textbf{Time (s)} \\
        \hline
        Baseline 1 & 85.2 & 0.83 & 12.5 \\
        Baseline 2 & 86.1 & 0.84 & 15.2 \\
        \textbf{Ours} & \textbf{88.5} & \textbf{0.87} & 13.8 \\
        \hline
    \end{tabular}
\end{table}

Include figures:

\begin{figure}[htbp]
    \centering
    % \includegraphics[width=0.45\textwidth]{figure.pdf}
    \fbox{\parbox{0.4\textwidth}{\centering\vspace{1.5cm}Figure Placeholder\vspace{1.5cm}}}
    \caption{Description of the figure.}
    \label{fig:results}
\end{figure}

\subsection{Analysis}

Analyze the results and discuss key findings.

%% ----------------------------------------------------------------------------
%% CONCLUSION
%% ----------------------------------------------------------------------------

\section{Conclusion}
\label{sec:conclusion}

Summarize contributions and main findings. Briefly mention limitations and
future work directions.

%% ----------------------------------------------------------------------------
%% ACKNOWLEDGMENT
%% ----------------------------------------------------------------------------

\section*{Acknowledgment}

This work was supported by [funding agency] under Grant No. [number].

%% ----------------------------------------------------------------------------
%% REFERENCES
%% ----------------------------------------------------------------------------

\begin{thebibliography}{00}
\bibitem{ref1}
A. Author, B. Author, and C. Author, ``Title of the paper,''
\textit{IEEE Trans. on Something}, vol. 1, no. 1, pp. 1--10, 2024.

\bibitem{ref2}
D. Author and E. Author, ``Another paper title,''
in \textit{Proc. IEEE Conference}, City, Country, 2024, pp. 100--105.
\end{thebibliography}

\end{document}
