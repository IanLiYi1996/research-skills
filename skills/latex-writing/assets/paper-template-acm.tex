% ACM Publication Template
% Research Skills Plugin - LaTeX Writing Skill
%
% Document class options:
%   manuscript - For review submission
%   acmsmall   - Small single column format
%   acmlarge   - Large single column format
%   acmtog     - For TOG journal articles
%   sigconf    - For SIGCONF proceedings
%   sigplan    - For SIGPLAN proceedings
%   sigchi     - For SIGCHI proceedings
%
% Compile with: pdflatex paper.tex && bibtex paper && pdflatex paper.tex && pdflatex paper.tex

\documentclass[sigconf,anonymous]{acmart}
% Use 'anonymous' for double-blind review, remove for camera-ready

%% ============================================================================
%% PACKAGES
%% ============================================================================

% Additional math support
\usepackage{amsmath,amssymb}

% Graphics
\usepackage{graphicx}

% Tables
\usepackage{booktabs}

% Algorithms
\usepackage{algorithm}
\usepackage{algorithmic}

% Subcaptions
\usepackage{subcaption}

%% ============================================================================
%% ACM METADATA
%% ============================================================================

% Copyright
\setcopyright{acmcopyright}
\copyrightyear{2024}
\acmYear{2024}
\acmDOI{XXXXXXX.XXXXXXX}

% Conference info
\acmConference[Conference'24]{ACM Conference Name}{Month Day--Day, 2024}{City, Country}
\acmBooktitle{Proceedings of ACM Conference Name (Conference'24), Month Day--Day, 2024, City, Country}
\acmPrice{15.00}
\acmISBN{978-1-4503-XXXX-X/24/XX}

%% ============================================================================
%% CUSTOM COMMANDS
%% ============================================================================

\newcommand{\R}{\mathbb{R}}
\newcommand{\E}{\mathbb{E}}
\newcommand{\argmax}{\operatorname{argmax}}
\newcommand{\argmin}{\operatorname{argmin}}

%% ============================================================================
%% DOCUMENT CONTENT
%% ============================================================================

\begin{document}

%% ----------------------------------------------------------------------------
%% TITLE AND AUTHORS
%% ----------------------------------------------------------------------------

\title{Your Paper Title Here}
\subtitle{Subtitle if needed}

\author{First Author}
\email{first@university.edu}
\orcid{0000-0000-0000-0000}
\affiliation{
    \institution{University Name}
    \department{Department of Computer Science}
    \city{City}
    \country{Country}
}

\author{Second Author}
\email{second@another.edu}
\affiliation{
    \institution{Another University}
    \department{Department of Computer Science}
    \city{City}
    \country{Country}
}

\author{Third Author}
\email{third@university.edu}
\affiliation{
    \institution{University Name}
    \department{Department of Computer Science}
    \city{City}
    \country{Country}
}

%% ----------------------------------------------------------------------------
%% ABSTRACT
%% ----------------------------------------------------------------------------

\begin{abstract}
    This template provides guidance for ACM publication formatting.
    Write a concise abstract of 150-250 words summarizing the problem,
    approach, key results, and conclusions. The abstract should be
    self-contained and not include citations.
\end{abstract}

%% ----------------------------------------------------------------------------
%% CCS CONCEPTS AND KEYWORDS
%% ----------------------------------------------------------------------------

% CCS Concepts - see https://dl.acm.org/ccs
\begin{CCSXML}
<ccs2012>
   <concept>
       <concept_id>10010147.10010178.10010179</concept_id>
       <concept_desc>Computing methodologies~Machine learning</concept_desc>
       <concept_significance>500</concept_significance>
   </concept>
</ccs2012>
\end{CCSXML}

\ccsdesc[500]{Computing methodologies~Machine learning}

\keywords{keyword1, keyword2, keyword3, keyword4}

%% ----------------------------------------------------------------------------
%% TITLE PAGE
%% ----------------------------------------------------------------------------

\maketitle

%% ----------------------------------------------------------------------------
%% INTRODUCTION
%% ----------------------------------------------------------------------------

\section{Introduction}
\label{sec:intro}

Introduce your research problem and motivate the work. ACM papers typically
follow this structure for the introduction:

\begin{itemize}
    \item Establish the context and importance of the research area
    \item Describe the specific problem being addressed
    \item Briefly review existing approaches and their limitations
    \item State your contributions clearly
    \item Outline the paper organization
\end{itemize}

Our main contributions are:
\begin{enumerate}
    \item First contribution
    \item Second contribution
    \item Third contribution
\end{enumerate}

%% ----------------------------------------------------------------------------
%% RELATED WORK
%% ----------------------------------------------------------------------------

\section{Related Work}
\label{sec:related}

Review relevant prior work. Use citations like \cite{example2024} to reference
related papers.

\subsection{Topic Area A}
Discuss related work in this area.

\subsection{Topic Area B}
Discuss related work in another area.

%% ----------------------------------------------------------------------------
%% METHODOLOGY
%% ----------------------------------------------------------------------------

\section{Methodology}
\label{sec:method}

Describe your approach in detail.

\subsection{Problem Definition}

Formally define the problem:
\begin{equation}
    \min_{\theta} \mathcal{L}(\theta) = \frac{1}{n} \sum_{i=1}^{n} \ell(f_\theta(x_i), y_i)
    \label{eq:objective}
\end{equation}

\subsection{Proposed Approach}

Describe your method. You can use algorithms:

\begin{algorithm}
\caption{Algorithm Name}
\label{alg:method}
\begin{algorithmic}[1]
    \STATE \textbf{Input:} Data $\mathcal{D}$, learning rate $\eta$
    \STATE Initialize parameters $\theta_0$
    \FOR{$t = 1$ to $T$}
        \STATE Compute gradient $g_t = \nabla \mathcal{L}(\theta_t)$
        \STATE Update $\theta_{t+1} = \theta_t - \eta g_t$
    \ENDFOR
    \STATE \textbf{Output:} Trained parameters $\theta_T$
\end{algorithmic}
\end{algorithm}

%% ----------------------------------------------------------------------------
%% EVALUATION
%% ----------------------------------------------------------------------------

\section{Evaluation}
\label{sec:eval}

Present your experimental evaluation.

\subsection{Experimental Setup}

Describe datasets, metrics, baselines, and implementation details.

\subsection{Results}

Present quantitative results:

\begin{table}[htbp]
    \caption{Comparison of methods on benchmark datasets.}
    \label{tab:results}
    \begin{tabular}{lcccc}
        \toprule
        Method & Dataset A & Dataset B & Dataset C & Avg \\
        \midrule
        Baseline 1 & 85.2 & 78.4 & 91.0 & 84.9 \\
        Baseline 2 & 86.1 & 79.2 & 91.5 & 85.6 \\
        \textbf{Ours} & \textbf{88.5} & \textbf{82.1} & \textbf{93.2} & \textbf{87.9} \\
        \bottomrule
    \end{tabular}
\end{table}

Include figures:

\begin{figure}[htbp]
    \centering
    % \includegraphics[width=0.9\columnwidth]{figure.pdf}
    \fbox{\parbox{0.85\columnwidth}{\centering\vspace{2cm}Figure Placeholder\vspace{2cm}}}
    \caption{Description of the figure and key observations.}
    \label{fig:results}
\end{figure}

\subsection{Analysis}

Provide detailed analysis of results, including ablation studies.

%% ----------------------------------------------------------------------------
%% DISCUSSION
%% ----------------------------------------------------------------------------

\section{Discussion}
\label{sec:discussion}

Discuss implications, limitations, and potential impact of your work.

\subsection{Limitations}

Acknowledge limitations of your approach.

\subsection{Future Work}

Suggest directions for future research.

%% ----------------------------------------------------------------------------
%% CONCLUSION
%% ----------------------------------------------------------------------------

\section{Conclusion}
\label{sec:conclusion}

Summarize your contributions and key findings. Restate the significance and
potential impact of your work.

%% ----------------------------------------------------------------------------
%% ACKNOWLEDGMENTS
%% ----------------------------------------------------------------------------

\begin{acks}
This work was supported by [funding agency] under Grant No. [number].
We thank the anonymous reviewers for their valuable feedback.
\end{acks}

%% ----------------------------------------------------------------------------
%% REFERENCES
%% ----------------------------------------------------------------------------

\bibliographystyle{ACM-Reference-Format}
% \bibliography{references}  % Uncomment when using .bib file

% Sample references (remove when using .bib file)
\begin{thebibliography}{9}
\bibitem{example2024}
Author, A., Author, B., \& Author, C. 2024.
Title of the paper.
In \textit{Proceedings of ACM Conference (Conference'24)}.
ACM, New York, NY, USA, 1--10.
\url{https://doi.org/10.1145/XXXXXXX.XXXXXXX}
\end{thebibliography}

%% ----------------------------------------------------------------------------
%% APPENDIX
%% ----------------------------------------------------------------------------

\appendix

\section{Supplementary Material}
\label{sec:appendix}

Include additional details, proofs, or extended results.

\end{document}
