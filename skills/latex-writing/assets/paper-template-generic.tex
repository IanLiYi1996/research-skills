% Generic Academic Paper Template
% Research Skills Plugin - LaTeX Writing Skill
%
% Usage: Copy this file and modify for your paper
% Compile with: pdflatex paper.tex && bibtex paper && pdflatex paper.tex && pdflatex paper.tex

\documentclass[11pt,a4paper]{article}

%% ============================================================================
%% PACKAGES
%% ============================================================================

% Page layout
\usepackage[margin=1in]{geometry}
\usepackage{setspace}
\onehalfspacing

% Mathematics
\usepackage{amsmath,amssymb,amsthm}
\usepackage{mathtools}

% Graphics and figures
\usepackage{graphicx}
\usepackage{float}
\usepackage{subcaption}

% Tables
\usepackage{booktabs}
\usepackage{array}
\usepackage{multirow}

% Colors and hyperlinks
\usepackage[dvipsnames]{xcolor}
\usepackage[colorlinks=true,linkcolor=blue,citecolor=blue,urlcolor=blue]{hyperref}

% Bibliography
\usepackage[numbers,sort&compress]{natbib}

% Code listings (optional)
\usepackage{listings}

% Units (optional)
\usepackage{siunitx}

% Clever references
\usepackage{cleveref}

% Typography improvements
\usepackage{microtype}

%% ============================================================================
%% CUSTOM COMMANDS
%% ============================================================================

% Math shortcuts
\newcommand{\R}{\mathbb{R}}
\newcommand{\N}{\mathbb{N}}
\newcommand{\E}{\mathbb{E}}
\newcommand{\Var}{\mathrm{Var}}
\newcommand{\argmax}{\operatorname{argmax}}
\newcommand{\argmin}{\operatorname{argmin}}

% Vector and matrix notation
\renewcommand{\vec}[1]{\mathbf{#1}}
\newcommand{\mat}[1]{\mathbf{#1}}

% Theorem environments
\theoremstyle{plain}
\newtheorem{theorem}{Theorem}[section]
\newtheorem{lemma}[theorem]{Lemma}
\newtheorem{proposition}[theorem]{Proposition}
\newtheorem{corollary}[theorem]{Corollary}

\theoremstyle{definition}
\newtheorem{definition}[theorem]{Definition}
\newtheorem{example}[theorem]{Example}

\theoremstyle{remark}
\newtheorem{remark}[theorem]{Remark}

%% ============================================================================
%% DOCUMENT METADATA
%% ============================================================================

\title{Your Paper Title Here}

\author{
    First Author\thanks{Corresponding author. Email: first@example.com} \\
    \small Department of Computer Science \\
    \small University Name \\
    \and
    Second Author \\
    \small Department of Mathematics \\
    \small Another University
}

\date{\today}

%% ============================================================================
%% DOCUMENT CONTENT
%% ============================================================================

\begin{document}

\maketitle

%% ----------------------------------------------------------------------------
%% ABSTRACT
%% ----------------------------------------------------------------------------

\begin{abstract}
    Write your abstract here. The abstract should be a concise summary of your
    paper, typically 150-250 words. It should include: (1) the problem or
    research question, (2) the methodology or approach, (3) key findings or
    results, and (4) main conclusions and implications.

    \textbf{Keywords:} keyword1, keyword2, keyword3, keyword4, keyword5
\end{abstract}

%% ----------------------------------------------------------------------------
%% INTRODUCTION
%% ----------------------------------------------------------------------------

\section{Introduction}
\label{sec:introduction}

Introduce your research topic and motivate the problem. This section should:
\begin{itemize}
    \item Establish the context and importance of the research area
    \item Review relevant background and prior work
    \item Identify the gap or problem your research addresses
    \item State your research objectives and contributions
    \item Outline the structure of the paper
\end{itemize}

The remainder of this paper is organized as follows. \Cref{sec:related} reviews
related work. \Cref{sec:method} describes our methodology. \Cref{sec:results}
presents experimental results. \Cref{sec:discussion} discusses the findings.
Finally, \Cref{sec:conclusion} concludes the paper.

%% ----------------------------------------------------------------------------
%% RELATED WORK
%% ----------------------------------------------------------------------------

\section{Related Work}
\label{sec:related}

Review the relevant literature and position your work within the field.

\subsection{Topic Area 1}

Discuss related work in the first topic area. Use citations like
\cite{example2024} to reference prior work.

\subsection{Topic Area 2}

Discuss related work in the second topic area.

%% ----------------------------------------------------------------------------
%% METHODOLOGY
%% ----------------------------------------------------------------------------

\section{Methodology}
\label{sec:method}

Describe your approach or methodology in detail.

\subsection{Problem Formulation}

Define the problem mathematically. For example:

\begin{definition}[Problem Name]
    Given a dataset $\mathcal{D} = \{(\vec{x}_i, y_i)\}_{i=1}^n$ where
    $\vec{x}_i \in \R^d$ and $y_i \in \{0, 1\}$, find the optimal parameters
    $\vec{\theta}^*$ that minimize:
    \begin{equation}
        \vec{\theta}^* = \argmin_{\vec{\theta}} \frac{1}{n} \sum_{i=1}^n
        \mathcal{L}(f(\vec{x}_i; \vec{\theta}), y_i)
        \label{eq:objective}
    \end{equation}
\end{definition}

\subsection{Proposed Approach}

Describe your proposed method. You can use aligned equations:

\begin{align}
    \nabla_\theta \mathcal{L} &= \frac{\partial \mathcal{L}}{\partial \theta}
    \label{eq:gradient} \\
    \theta_{t+1} &= \theta_t - \eta \nabla_\theta \mathcal{L}
    \label{eq:update}
\end{align}

%% ----------------------------------------------------------------------------
%% RESULTS
%% ----------------------------------------------------------------------------

\section{Experimental Results}
\label{sec:results}

Present your experimental setup and results.

\subsection{Experimental Setup}

Describe datasets, evaluation metrics, and baselines.

\subsection{Main Results}

Present your main results. Use tables for quantitative comparisons:

\begin{table}[htbp]
    \centering
    \caption{Comparison of methods on benchmark datasets. Best results in
    \textbf{bold}.}
    \label{tab:results}
    \begin{tabular}{lccc}
        \toprule
        Method & Dataset A & Dataset B & Dataset C \\
        \midrule
        Baseline 1 & 85.2 & 78.4 & 91.0 \\
        Baseline 2 & 86.1 & 79.2 & 91.5 \\
        Our Method & \textbf{88.5} & \textbf{82.1} & \textbf{93.2} \\
        \bottomrule
    \end{tabular}
\end{table}

Include figures for visual results:

\begin{figure}[htbp]
    \centering
    % \includegraphics[width=0.8\textwidth]{figures/results.pdf}
    \fbox{\parbox{0.8\textwidth}{\centering\vspace{2cm}Figure Placeholder\vspace{2cm}}}
    \caption{Visualization of experimental results. (a) Subplot description.
    (b) Another subplot description.}
    \label{fig:results}
\end{figure}

\subsection{Ablation Study}

Analyze the contribution of different components.

%% ----------------------------------------------------------------------------
%% DISCUSSION
%% ----------------------------------------------------------------------------

\section{Discussion}
\label{sec:discussion}

Interpret your results and discuss their implications.

\subsection{Key Findings}

Summarize the main findings from your experiments.

\subsection{Limitations}

Acknowledge limitations of your approach.

\subsection{Future Work}

Suggest directions for future research.

%% ----------------------------------------------------------------------------
%% CONCLUSION
%% ----------------------------------------------------------------------------

\section{Conclusion}
\label{sec:conclusion}

Summarize your contributions and main findings. Restate the significance of
your work and its potential impact.

%% ----------------------------------------------------------------------------
%% ACKNOWLEDGMENTS
%% ----------------------------------------------------------------------------

\section*{Acknowledgments}

Thank funding agencies, collaborators, and reviewers. For example:
This work was supported by [Grant Name] under Grant No. [Number].

%% ----------------------------------------------------------------------------
%% BIBLIOGRAPHY
%% ----------------------------------------------------------------------------

\bibliographystyle{plainnat}
% \bibliography{references}  % Uncomment and create references.bib

% Sample bibliography entries (remove when using .bib file)
\begin{thebibliography}{9}
\bibitem{example2024}
Author, A., \& Author, B. (2024).
\newblock Title of the paper.
\newblock \emph{Journal Name}, 1(1), 1--10.
\end{thebibliography}

%% ----------------------------------------------------------------------------
%% APPENDIX
%% ----------------------------------------------------------------------------

\appendix

\section{Supplementary Material}
\label{sec:appendix}

Include additional details, proofs, or extended results here.

\begin{theorem}[Example Theorem]
    For all $n \in \N$, we have $\sum_{i=1}^n i = \frac{n(n+1)}{2}$.
\end{theorem}

\begin{proof}
    By induction on $n$. The base case $n=1$ is trivial. For the inductive
    step, assume the result holds for $n$. Then for $n+1$:
    \begin{align*}
        \sum_{i=1}^{n+1} i &= \sum_{i=1}^n i + (n+1) \\
        &= \frac{n(n+1)}{2} + (n+1) \\
        &= \frac{(n+1)(n+2)}{2}
    \end{align*}
    which completes the proof.
\end{proof}

\end{document}
